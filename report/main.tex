\documentclass[conference]{IEEEtran}
\usepackage{cite}
\usepackage{amsmath,amssymb,amsfonts}
\usepackage{algorithmic}
\usepackage{graphicx}
\usepackage{textcomp}
\usepackage{xcolor}
\def\BibTeX{{\rm B\kern-.05em{\sc i\kern-.025em b}\kern-.08em
    T\kern-.1667em\lower.7ex\hbox{E}\kern-.125emX}}


\begin{document}

\title{CMP9135M | Computer Vision}

\author{\IEEEauthorblockN{1\textsuperscript{st} George Davies}
    \IEEEauthorblockA{
        \textit{School of Computer Science} \\
        \textit{University of Lincoln}\\
        Lincoln, United Kingdom \\
        27421138@students.lincoln.ac.uk
    }
}

\maketitle

\begin{abstract}
\end{abstract}

\begin{IEEEkeywords}
\end{IEEEkeywords}

\section*{Task 1 | Image Processing}
%Download two files: ‘ball_frames.zip’ from Blackboard. Unzip the dataset file, you should obtain a set of 126 images. 
%Among those images, there are 63 ball colour images and 63 corresponding ball mask images (ground-truth segmentation). 
%Figure 1 shows an example of one ball image and its corresponding mask image. Please use conventional computer vision 
%techniques (no deep/machine learning solution allowed in this task) to implement the following tasks. Please note that
% you are expected to develop one model with same parameter settings for all the images.

\subsection*{Task 1.a | Automated ball objects segmentation}
%Automated ball objects segmentation. For each image, automatically segment the balls from background.

\subsection*{Task 1.b | Segmentation evaluation}
%Segmentation evaluation. For each ball image, calculate the Dice Similarity Score (DS) which is defined in Equation 1;
%where M is the segmented ball region you obtained from Task 1, and S is the corresponding ground-truth binary ball 
%mask. Please note that, in this case, for the provided ball mask images, you can convert the grayscale images into 
%binary images (e.g. ball object and background), and use the converted binary images as ground-truth mask.

%The calculated DS shall be between 0 and 1. For example, DS is 1 if your segmentation matches perfectly with the 
%ground-truth mask, whist DS is 0 if there is no overlap between your segmentation and ground-truth mask.

%Your report should include: 1) for all the 63 ball images, please provide a bar graph with x-axis representing the 
%number of the image, and y-axis representing the corresponding DS. 2) calculate the mean and standard deviation of the
%DS for all the 63 images, and 3) briefly describe and justify the implementation steps. Please note that you are 
%required to show the best 5 and worst 5 segmented ball images (along with the corresponding ball GT mask images) in 
%the Appendix.


\begin{figure}[htbp]
    \centering
    \includegraphics[width=\columnwidth]{figures/DS_bar_graph.pdf}
    \caption{Dice Similarity score for all 63 images\label{fig:DS_bar_graph}}
\end{figure}

\section*{Task 2 | Feature Calculation}
%This part of the assignment will deal with feature extraction, more specifically you will be examining texture and 
%shape features. Using the provided GT ball masks to obtain the corresponding ball patches from original RGB images, 
%carrying out the following tasks.

\subsection*{Task 2.a | Shape features}
%For each of the ball patches, calculate four different shape features discussed in the lectures (solidity, 
%non-compactness, circularity, eccentricity). Plot the distribution of all the four features, per ball type.

\begin{figure}[htbp]
    \centering
    \includegraphics[width=\columnwidth]{figures/shape_feats.pdf}
    \caption{Shape features\label{fig:shape_feats}}
\end{figure}


\subsection*{Task 2.b | Texture features}
%Calculate the normalised grey-level co-occurrence matrix in four orientations (0°, 45°, 90°, 135°) for the patches 
%from the three balls, separately for each of the colour channels (red, green, blue). For each orientation, calculate 
%the first three features proposed by Haralick et al. (Angular Second Moment, Contrast, Correlation), and produce 
%per-patch features by calculating the feature average and range across the 4 orientations. Select one feature from 
%each of the colour channels and plot the distribution per ball type.

\begin{figure}[htbp]
    \centering
    \includegraphics[width=\columnwidth]{figures/averages.pdf}
    \caption{Texture features Averages\label{fig:tex_feats_avgs}}
\end{figure}

\begin{figure}[htbp]
    \centering
    \includegraphics[width=\columnwidth]{figures/ranges.pdf}
    \caption{Texture features Ranges\label{fig:tex_feats_ranges}}
\end{figure}


\subsection*{Task 2.c | Discriminative information}
%Based on your visualisations in part a) and b), discuss which features appear to be best at differentiating between 
%different ball types. For each ball type, are shape or texture features more informative? Which ball type is the 
%easiest/hardest to distinguish, based on the calculated features? Which other features or types of features would you 
%suggest for the task of differentiating between the different ball types and why?
%Analyse and discuss your findings in the report.


\section*{Task 3 | Object Tracking}
%Download from Blackboard the data files 'x.csv' and 'y.csv', which contain the real coordinates [x,y] of a moving 
%ball, and the files 'na.csv' and 'nb.csv', which contain their noisy version provided by some segmentation and 
%recognition for the football (e.g. frame-to-frame image segmentation of the target). Implement a Kalman filter from 
%scratch (not using any method/class from pre-built libraries) that accepts as input the noisy coordinates [na,nb] and 
%produces as output the estimated coordinates [x*,y*]. For this, you should use a Constant Velocity motion model F with 
%constant time intervals Δt = 0.5 and a Cartesian observation model H.

\subsection*{Task 3.a | Kalman filter tracking}
%You should plot the estimated trajectory of coordinates [x*,y*], together with the real [x,y] and the noisy ones [a,b] 
%for comparison. Discuss how you arrive to the solution.

\begin{figure}[htbp]
    \centering
    \includegraphics[width=\columnwidth]{figures/kalman.pdf}
    \caption{plot of the estimated trajectory of coordinates $[x_, y_]$, together with the real $[x,y]$ and the noisy $[na, nb]$ for comparison\label{fig:kalman}}
\end{figure}

\subsection*{Task 3.b | Evaluation}
%You should also assess the quality of the tracking by calculating the mean and standard deviation of the Root Mean 
%Squared error (include the mathematical formulas you used for the error calculation in your report). Compare both 
%noisy and estimated coordinates to the ground truth. Adjust the parameters associated with the Kalman filter, justify
% any choices of parameter(s) associated with Kalman Filter that can give you better estimation of the coordinates that
% are closer to the ground truth. Discuss and justify your findings in the report.

\appendix

\begin{figure}[htbp]
    \centering
    \includegraphics[width=\columnwidth]{figures/best.pdf}
    \caption{Best 5 segmented ball images compared to the ground truth\label{apx:best}}
\end{figure}
\begin{figure}[htbp]
    \centering
    \includegraphics[width=\columnwidth]{figures/worst.pdf}
    \caption{Worst 5 segmented ball images compared to the ground truth\label{apx:worst}}
\end{figure}


\end{document}
